% Tipo de letra Times New Roman, 12 puntos y tamaño carta
    \documentclass[12pt,letterpaper,openany,oneside]{book}
    \usepackage{natbib}
    \bibliographystyle{apalike2}
    \usepackage[T1]{fontenc}
    \usepackage{times}
    \usepackage[utf8]{inputenc}
    \usepackage[spanish]{babel}
    \renewcommand{\baselinestretch}{1.6}
%Margen superior 4cm, inferior 2.5 cm, margen izquierdo 4 cm y derecho 2.5 cm para todo el documento
    \usepackage[left=4cm,right=2.5cm,top=4cm,bottom=2.5cm]{geometry}


\usepackage{xcolor}
    \definecolor{azul}{rgb}{0.17,0.49,0.69}
    \definecolor{gris}{gray}{0.875}
\usepackage{rotating}
\usepackage{titlesec}
\usepackage{hyperref}
\hypersetup{
	colorlinks=true,
	linkcolor=black,
	urlcolor=blue,
	citecolor=azul,
}

%----------------Contorno
\usepackage{fancyhdr}
\pagestyle{fancy}
\fancyhf{}
	\chead[]{}
	\lhead[]{}
	\rhead[]{\leftmark}
	\lfoot[]{}
	\rfoot[]{\thepage}
	\cfoot[]{}




\usepackage{appendix}
\usepackage{lipsum}


\begin{document}
%Estructura del documento
	%----------------------------------PRELIMINARES
	\frontmatter
	\begin{titlepage}
	\begin{center}
		{\LARGE UNIVERSIDAD MAYOR DE SAN ANDRÉS}
		
		{\Large FACULTAD DE INGENIERÍA}
		
		{\Large CARRERA DE INGENIERÍA INDUSTRIAL}
		\vspace{1cm}%Añade un espacio de 1cm 
		
		%El logo con las dimensiones correctas las puede descargar en: https://www.dropbox.com/s/mgn9wn8p8q8n5k4/Logo_UMSA.png?dl=0
		\includegraphics[scale=1]{Imagenes/Logo_UMSA.png} 
		
	    \vspace{1cm} 
		{\Large TÍTULO DEL PROYECTO}
	    \vspace{1cm} 
	    
		{\small Proyecto de grado presentado para la obtención del Grado de Licenciatura}
	
		{\Large POR: NOMBRE DE LA AUTORA O AUTOR}
		
		\vspace{1cm} 
		{\large TUTOR: NOMBRE DE LA TUTORA O TUTOR}
        \vspace{1cm} 

		\vfill
		LA PAZ-BOLIVIA
		
	    Julio, 2021
	\end{center}
\end{titlepage}
	\begin{center}
	{\large UNIVERSIDAD MAYOR DE SAN ANDRÉS}

	{\large FACULTAD DE INGENIERÍA}

	{\large CARRERA DE INGENIERÍA INDUSTRIAL}
\end{center}
Proyecto de grado:
\begin{center}
	\vspace{1cm}
	{\Large Titulo del Proyecto de Grado}
	\vspace{1cm}
\end{center}
Presentada por: Univ. Nombre de la autora o autor

Para optar el grado académico de \textbf{\textit{Licenciatura en Ingeniería Industrial}}

Nota numeral :......................................................

Nota literal :..........................................................

Ha sido..................................................................


Director de la carrera de Ingeniería Industrial: Ing. Franz Zenteno Benitez

\vspace{0.5cm} 
Tutor/Tutora: Ing.

\vspace{0.5cm} 
Tribunal: Ing.

\vspace{0.5cm} 
Tribunal: Ing. 

\vspace{0.5cm} 
Tribunal: Ing. 

\vspace{0.5cm} 
Tribunal: Ing.
	\chapter*{DEDICATORIA}
		\begin{center}
			\textit{Párrafo 1}	
		\end{center}
	
		\begin{center}
			\textit{Párrafo 2}
		\end{center}
	
		\begin{center}		
			\textit{Párrafo 3}
		\end{center}
	
		\begin{center}
			\textit{Párrafo 4}
		\end{center}
	\chapter*{AGRADECIMIENTOS}
    Parrafos del agradecimiento
    %Tabla de contenido
	    \tableofcontents
	%Índice de figuras
		\listoffigures
	%Índice de tablas
		\listoftables
	%Resumen en español e ingles y palabras clave
	    % Comando para palabras clave
        \providecommand{\palabrasclave}[1]
        {
          \small	
          \textbf{\textit{Palabras clave:}} #1
        }   
    % Keywords command
        \providecommand{\keywords}[1]
        {
          \small	
          \textbf{\textit{Keywords:}} #1
        }

\chapter*{RESUMEN}
    \palabrasclave{Palabras}
\chapter*{ABSTRACT}
    
    \keywords{Keywords}

	
	%------------------------------------TEXTO
	\mainmatter
	
	\chapter{INTRODUCCIÓN}\label{cap:Introduccion}
   Ejemplo \citep{fransoo2017reaching}
	\chapter{REVISIÓN DE LA LITERATURA}\label{cap:RevisionLiteratura}
    \section{Sección 1}
    \section{Sección 2}
    \section{Brechas y contribuciones}
	\chapter{METODOLOGÍA Y DATOS}\label{cap:MetodologiaDatos}
	\chapter{ANÁLISIS Y RESULTADOS}\label{cap:AnalisisResultados}
	\chapter{CONCLUSIONES, PERSPECTIVAS GERENCIALES E INVESTIGACIÓN FUTURA}\label{cap:Conclusiones}
    \section{Conclusiones}
    \section{Perspectivas Gerenciales}
    \section{Investigación Futura}
	
	%------------------------------------REFERENCIAS--------------------------------
	\backmatter
    \bibliography{Bibliografia}
	\addcontentsline{toc}{chapter}{BIBLIOGRAFÍA}
	\appendix
	\chapter*{APÉNDICES}
\addcontentsline{toc}{chapter}{APÉNDICES}

\end{document}